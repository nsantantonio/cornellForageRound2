\pagenumbering{gobble}
\documentclass[11pt, letterpaper]{moderncv}
\usepackage{times}
\usepackage{amsmath}


\usepackage[mmddyyyy,hhmmss]{datetime}
\usepackage{xcolor}

\definecolor{nicholasCol}{RGB}{203,97,63}
\newcommand{\nicholas}[1]{{\color{nicholasCol} [\textbf{NS:} #1 (\today\ \currenttime)]}}

\moderncvstyle{classic}
\moderncvcolor{green}
\usepackage[utf8]{inputenc} 
\usepackage[scale=0.75, margin = 0.75in]{geometry}
\geometry{top  = 20mm}

\name{Nicholas}{Santantonio}
\address{312 Bradfield Hall, Ithaca, NY 14853}{(505)~412-2738}
\email{ns722@cornell.edu}                               
\begin{document}
\recipient{Plant Breeding and Genetics Section}{School of Integrative Plant Sciences\\College of Agriculture and Life Sciences\\Cornell University\\240 Emerson Hall, Ithaca, NY 14853}
\date{\today}
\opening{Dear Hiring Committee,}
\closing{Respectfully yours,}
\makelettertitle


I am eager to apply to the Breeding for Sustainable Cropping Systems Assistant Professor position at Cornell University. I have combined a strong applied background in forage and small grains breeding programs with theoretical quantitative genetics, enabling me to integrate the newest computational technologies into the successful, working forage breeding program.

I have demonstrated commitment and determination in research, teaching and leadership. I obtained extra-mural funding to pursue the integration of digital ag and population-based genomic selection strategies for alfalfa improvement. To further my teaching experience and hone my teaching philosophy, I co-instructed PLBRG7420 this past fall. Serving as a co-leader for the Diversity Preview Weekend, I have shown a dedication to learning about, and working toward a diverse, equitable and inclusive academic environment. 

Breeding for sustainable cropping systems will require expedited adaptation to reduced agronomic inputs and marginal lands. By incorporating genome-wide information and proximal sensing, I intend to shift from the traditional $20^\text{th}$ century breeding program to a data-driven $21^\text{st}$ century breeding model. This transition will provide a valuable resource for public outreach, where farmers and consumers can learn how we are adapting the latest technologies to help fortify our food systems.

Moving forward, I intend to develop relationships with farmers in New York to better understand how the forage breeding program can effectively serve their need to sustain their farms and dairies. I aim to help prepare future students for data-driven plant breeding by teaching courses with a quantitative, hands on approach. Most importantly, I intend to pursue several initiatives to develop community based outreach programs, create a diverse experience requirement for graduate students, and shed light on the hidden curriculum in academia. I am dedicated to cultivating a safe, inclusive and equitable environment where students and staff of all backgrounds can thrive. 

I implore you to carefully consider my research and teaching merits, as well as my contribution to diversity, equity and inclusion (DEI). In the year since I last applied to this position, I obtained a grant to pursue research in alfalfa, collaborated with international partners, submitted two first author publications from my postdoc, co-instructed a PhD course in advanced quantitative genetics and pursued learning and engaging in DEI initiatives. 

I would like to thank the hiring committee for considering my application to the Breeding for Sustainable Cropping Systems position. Under the leadership of Professors Murphy and Viands, the Cornell forage breeding program has a rich legacy. I intend to preserve the valuable forage genetic resources the program has developed over the last century, while also adapting the program to the latest technologies and new goals in sustainable agriculture.

\bigskip

Sincerely,\\
\includegraphics[height=2cm]{NSsigWhiteBG}\\
Nicholas Santantonio

\end{document}


\makeletterclosing

\end{document}

