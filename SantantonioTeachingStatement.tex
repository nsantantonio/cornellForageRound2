\documentclass[11pt]{article}
\usepackage{amsmath}
\usepackage{amssymb}
\usepackage[backend=bibtex, style=authoryear]{biblatex}

% \addbibresource{~/Dropbox/TAGreview/PhDreferences.bib}

\textwidth=470pt
\oddsidemargin=0pt
\topmargin=0pt
\headheight=0pt
\textheight=650pt
\headsep=0pt

\pdfpagewidth=\paperwidth
\pdfpageheight=\paperheight

\title{Statement of Research}


\author{Nicholas Santantonio}
\date{\today}


\newcommand{\gxe}{G$\times$E}
\newcommand{\gxg}{G$\times$G}


\begin{document}

\section*{\centering Statement of Teaching}
\begin{center} Nicholas Santantonio \end{center}
% \subsection*{Introduction}

\noindent Plant breeders have traditionally been generalists, combining genetics with a range of plant sciences to identify farmers' needs and turn out products to meet those needs. However, the range of skills required in the field is rapidly increasing. Proficiency in statistics, programming, bioinformatics and computational biology are now expected in addition to the traditional skills in physiology, pathology, agronomy and genetics. As educators, we must decide how to update our curriculum to best prepare students for careers in the new era of digital agriculture. 


Both generalists and specialists will be needed for integrating genomics and digital agriculture into modern breeding programs. Generalists may require instruction in team management, whereas specialists may need more rigorous courses in areas previously considered outside plant breeding, such as engineering and computer science. Specialists will be vital for collecting and storing large amounts of data that will be used to model and predict crop growth and development. Generalists will coordinate those activities to lead a team that produces new varieties to meet farmers' needs. 


The widening breadth of the plant breeding discipline may demand multiple paths of instruction. While I believe all future scientists will require some graduate level statistics and computational coursework, a more structured series of quantitative courses would benefit students seeking specialization. Students who chose this path would finish graduate school with the comprehension and ability to effectively adapt the latest computational techniques to meet future breeding goals.%in plant breeding, including the use of genome-wide information for decision making.


% Due to the wide breadth of the plant breeding discipline, it may be pertinent to develop two or three subdiscipline paths of instruction. Currently, most plant breeding students do not acquire in-depth statistics and programming skills until graduate school, impeding their progress while they learn to grapple with these new languages. While I believe all future plant breeders require some degree of graduate level statistics and computational coursework, a more structured series of quantitative and computational courses would benefit those students seeking specialization in these growing areas. I would like to work with the quantitatively oriented faculty within plant breeding and outside the section in biometry and computational biology to build a quantitative/computational genetics graduate curriculum. Lower level courses should be aimed at recruiting students' interests, while more advanced courses would be targeted to those who choose to specialize.

% At the undergraduate level, plant breeding courses at Cornell are rather limited, with only three courses below the 4000 level. Heightened interest in food systems has drawn many young people to seek out information about how crops have been genetically manipulated by humans. An important goal of the PLBRG2010 course is to pique interest and motivate students to pursue plant breeding as a profession. However, there is relatively little infrastructure to foster such an interest once it is seeded. A stronger undergraduate curriculum in new and traditional plant breeding technologies would increase enrollment in the Plant Breeding minor and add to the Section's value to the university.


\subsection*{Teaching Philosophy}

Inclusive teaching tactics are key to ensuring equal access to knowledge in the classroom. Expectations must be made clear, and reinforced throughout the semester so that students do not get behind and fail to meet milestones. It is important for students to be exposed to plant breeding ideas from multiple perspectives, and they must be given the opportunity to demonstrate critical thinking on different levels. While some students may show analytical thinking and synthesis on exams, others may shine in more hands-on projects. Most mathematical and computational learning occurs through doing, not through watching. The lecture is important to present material in a concise structured manner, but concepts are cemented when the student can reconstruct the ideas on their own time.

% Longer term projects provide students the opportunity to practice and apply concepts in a more autonomous environment, and are invaluable for assessing comprehension and critical thinking. In addition, regularly assigned homework and hands-on labs are crucial for estimation of the pace and overall understanding of the material presented, such that adjustments can be made where necessary. The greenhouse and field facilities on campus provide opportunities to get students out of the classroom to see plant breeding in practice. Public databases provide resources where students can get experience working with real, and often large, datasets. Computer simulations are useful tools to evaluate comprehension, where in order to simulate a system correctly, the student must understand that system well.

Longer term projects provide students the opportunity to practice and apply concepts in a more autonomous environment, and are invaluable for assessing comprehension and critical thinking. In addition, regularly assigned homework and hands-on labs are crucial for evaluation of the pace and overall understanding of the material presented. The greenhouse and field facilities on campus provide opportunities to get students out of the classroom to see plant breeding in practice. Public databases provide resources where students can get experience working with real datasets. Computer simulations are useful tools to evaluate comprehension, where in order to simulate a system correctly, the student must understand that system well.

All courses I instruct would contain a term project of relevant complexity to augment exams, in-class labs and homework assignments. For quantitatively oriented courses, these projects would be computational in nature, where students would use real or simulated data to explore the ideas covered in the course. They would then be asked to present their results in written and oral formats that mirror typical scientific communication. Projects may be team oriented to promote collaborative skills and project management. %For courses without a significant quantitative aspect, term projects may be formulated as a research proposal.

% For highly advanced courses, students could be evaluated using a gain in understanding approach. This approach would require students to submit responses to questions developed to guide understanding of the reading materials prior to class, with the ability to resubmit revised answers after attending. Concepts and questions would be sufficiently complex that prior effort and in class learning can be assessed.

I also intend to incoporate perspectives on how domestication, selection and seed systems work in different cultures through guest lectures and case studies. Reveiw of how seed systems function in other cultures will cultivate discussion on the benefits and drawbacks of these systems relative to the US and western European systems. 

\subsection*{Courses}

I intend target $21^\text{st}$ century plant breeding concepts for the Genetic Improvement of Crop Plants course (PLBRG4030), with focus on the use of genome-wide information to drive breeding decisions. The course would include a hands-on computational component to reflect the skills currently desired in the field. In addition, I intend to develop a graduate course to augment PLBRG7170, that would be held in the alternate year, as well as continue to co-instruct PLBRG7420. 

% would target $21^\text{st}$ century plant breeding concepts, focusing on the use of genome-wide information to drive breeding decisions. Starting with basic probability theory and the single locus model, the course would advance through genome-wide association, genomic prediction and selection, as well as selection theory and breeding program optimization. Concepts introduced in class would be reinforced through weekly computational labs and assignments that require students to write their own software to solve computational plant breeding problems. 

For PLBRG4030, computational labs would be used to augment student understanding of course material. Students would use available computational tools to analyze small example datasets with a focus on interpretation of results. For the term-project, students will be split into groups, and given a breeding senario and a dataset. They will determine the genetic architecture of their trait and develop a breeding strategy based on their senario and what they can learn from the data. By the end of the semester, students would be able to demonstrate critical thinking of plant breeding methods and ideas, with the ability to synthesize when given new plant systems or breeding goals. 


The graduate course topic is flexible to the needs of the section, however, it is my observation that an introductory quantitative genetics course would be useful to help prepare students for further quantitative study. This course might also provide some necessary skills to those who choose not to follow a quantitative path, without the rigorous mathematical detail of the PLBRG7170 quantitative genetics course. I would work with other quantitative faculty to build a course that augments other courses current offered, with little overlap in content. 


Concepts introduced in class would be reinforced through weekly computational labs and assignments that require students to write their own software to solve computational plant breeding problems. Assignments would be required to be submitted as typed documents in Markdown, \LaTeX\ or similar format, to expose students to more effective modes of mathematical communication outside of Microsoft Office. A term project might then consist of groups of 2-4 students finding a genotype-phenotype dataset, and working to develop a genotype to phenotype map, assess genomic predictability and construct a breeding scheme throughout the semester. 


\subsection*{Curriculum}

Currently, most life science students do not acquire in-depth statistics and programming skills until graduate school, impeding their progress while they learn to grapple with these new languages. Moving forward, I would like to work with faculty in CSES, statistics, engineering and computer science to build a quantitative/computational genetics undergraduate curriculum. Genetics is a vast field; it is imperative that students get exposure to and training in the rapidly changing environment in which they will soon be seeking jobs. While ``single-gene'' genetics is still an important field, more and more the community is focused on the complex ``omics'' network that is the foundation of complex organisms. Students in essentially all sub-fields of genetics will need to be able to deal with large datasets, using tailored algorithms to make inferences, predict the unobserved, and guide decision making. Dealing with big data requires skills in programming, linear algebra, statistics and machine learning. 


% \medskip

Genomics and digital agriculture are only just starting to change the landscape of food production. Quantitative skills are one of the specializations imperative in plant breeding, and Cornell must be at the forefront of best preparing the individuals who will usher in this new era.

\end{document}
