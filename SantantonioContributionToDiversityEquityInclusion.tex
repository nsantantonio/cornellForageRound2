\documentclass[11pt]{article}
\usepackage{amsmath}
\usepackage{amssymb}
\usepackage{hyperref}
\usepackage[backend=bibtex, style=authoryear]{biblatex}


\usepackage[mmddyyyy,hhmmss]{datetime}
\usepackage{xcolor}

\definecolor{nicholasCol}{RGB}{203,97,63}
\newcommand{\nicholas}[1]{{\color{nicholasCol} [\textbf{NS:} #1 (\today\ \currenttime)]}}

% \addbibresource{~/Dropbox/TAGreview/PhDreferences.bib}

\textwidth=470pt
\oddsidemargin=0pt
\topmargin=0pt
\headheight=0pt
\textheight=650pt
\headsep=0pt

\pdfpagewidth=\paperwidth
\pdfpageheight=\paperheight

\title{Contribution to Diversity Equity and Inclusion}


\author{Nicholas Santantonio}
\date{\today}


\newcommand{\gxe}{G$\times$E}
\newcommand{\gxg}{G$\times$G}


\begin{document}

\section*{\centering Statement of Diversity, Equity and Inclusion}
\begin{center} Nicholas Santantonio \end{center}
% \subsection*{Introduction}

% Need to say something about me/what I have done in the past. High school dropout? 

% high school dropout, dead end jobs, . know what it feels like to be lost.


% Can high potential be found by looking at grade variability. We use the mean (GPA), but this statistic only describes the center of the distribtion, not its breadth or shape. Hypothesis: students with highly variable grades are more likely to suceed than students with less variable grades given they have the same GPA. 

Everyone deserves an equal chance to show their potential. Often, it takes the right time, setting, person or simply an opportunity to inspire someone to set out on a path that will lead them to success and allow them to contribute in ways they never thought possible. It took several tries for me. Working minimum wage jobs for two years after dropping out of high school encouraged me to improve my life through education at New Mexico State University (NMSU). But even there I struggled, failing to meet a 2.0 GPA by the end of my fourth semester. With the help of several professors at NMSU and continued support from my family, I found my passion for plant breeding and genetics during my third year and everything changed.

I am one of the privileged ones. Without that support system, I may not have made it far enough to find that spark. So I ask myself, how can I help provide adequate support and opportunity to help create that spark in those far less privileged than I? As an educator, I can work to create opportunities for students from underrepresented groups to pursue further education and expose students to different cultural perspectives in and outside of the classroom. I can also take active measures to include students as part of a healthy lab culture, where expectations are clear and everyone is given equal opportunity to succeed in their own pursuits. Importantly, I can continue my own education into diversity, equity and inclusion (DE\&I) so that I can adapt my own efforts to better serve the community. 
% NMSU is a minority serving institution. Mentored high school student during my Masters.

% On more than one occasion I have found myself using my own experience with being lost to explain to a concerned parent that all is not indeed lost. I encouraged them to continue to support their child because you never know when that spark might occur. 

\section{The Diversity Preview Weekend}

To engage with the community and apply my skills to help the DE\&I effort, I joined the Diversity Preview Weekend (DPW, \href{www.cornelldpw.org}{www.cornelldpw.org}) initiative as a co-leader in September 2019, currently serving as the fund raising chair. DPW aims to increase DE\&I in STEM graduate programs at Cornell and around the US by inviting applicants from underrepresented groups to attend a fully-funded weekend at Cornell with the primary goal of empowering participants through workshops, campus tours, and familiarization with the application process. 

To reduce the financial burden on the participating departments of DPW, I have been working with several co-leaders to obtain outside funding sources. I put together an infographic (available upon request) which we used to describe what groups participants identify with and what the matriculation rates are, which was used in part to solicit Corteva, who agreed to sponsor 2 students to attend DPW. I am currently working to develop a pamphlet to encourage donors to join our effort by sponsoring students to attend DPW. 

To demonstrate the efficacy of the DPW program (now that data on participant outcomes is becoming available), and increase the exposure and outreach of DPW, I initiated \nicholas{remove initiated?} and then partnered with several DPW co-leaders to submit an abstract for a talk at the DE\&I section of the Ethical, Legal, Social Issues theme at The Allied Genetics Conference (TAGC, \href{genetics-gsa.org/tagc-2020/}{genetics-gsa.org/tagc-2020/}). Our hope is that this talk, to be delivered by a graduate student, Andrea Darby, of Entomology, can target faculty, staff and students to encourage undergraduates at their own institutions to apply for DPW, or even create similar programs at their own institutions. We also hope to solidify current donors, as well as attract new donors by demonstrating that DPW like programs are effective and that there is interest in such programs in the academic community.

\section{The Hidden Curriculum}

I recently attended the BTI Postgraduate Society (PGS) workshop ``Future Leaders in Plant Science: the Hidden Curriculum'', led by Janani Hariharan (see \href{science.sciencemag.org/content/364/6441/702.full}{Hariharan, Science 2019}). Hariharan demonstrated how social and professional norms to some are not made clear to others, through a role-playing group construction project in which only certain objectives were made clear to some actors. Playing the part of the assistant who is supposed to help without asking questions or knowledge of the project objectives, the exercise had a profound effect on me, shedding light on how frustrating it is to be in the dark when others expect you to conform/perform. 

There is a disconnect [and a large degree of uncertainty] for many students between what they think is expected of them, and what is actually expected of them. Some students who are more ``familiar'' with the US education system learn many ``untaught'' concepts that are not necessarily obvious or made accessible to those who are less familiar. For example, many students from underrepresented backgrounds, international students, and \emph{especially} visiting scientists, are given little guidance or resources outside of a desk to sit at. I believe there is almost a reinforcement of this process, where there is an unspoken attitude that ``the best ones will figure it out''. Such a sink or swim policy may identity a few good swimmers who already know how to swim, but it allows future [potential] Olympians to drown before they have ever been taught that they \emph{can} swim. 

As a graduate students and postdoc, I have seen this play out many times.

To help uncover this curriculum, I would like build a requirement for a negotiation process to define clearly written expectations should be negotiated by the committee and the student not only at onboarding, but on a yearly or semi-yearly basis. Expectations for faculty should also be discussed and 

From making students and postgrads feel comfortable at the table, to providing a safe place to ask academic and non-academic questions

a workshop on the hidden curriculum that emphasized the difficulties and inequalities created when  

How to apply to jobs, negotiate a salary, , even how to write a diversity, equity and inclusion statement. I wasn't shown how to meet any of these expectations. I am learning them through trial and, so far, error. I want to incoporate these skills into the written curriculum, by learning what is expected, and . [Amy: dont talk about yourself. experienced by many/most]. 

 I am continuing to learn how I can contribute, and 

and addressed ways to mitigate, or uncover this curriculum.

% An issue I have observed as a relatively recent graduate student is the existence of a hidden curriculum. 




\section{Diversity Inclusion and Equity through experience}

I taken implicit bias assessments to gauge what my own biases are.

It is imperative that we as educators provide an inclusive environment that strives to make resources equally available, expectations clear and readily checks in to ensure that individuals have not strayed to far from a path toward success. Toward this end, I want to work to develop a series of DE\&I related requirements for SIPS faculty, staff and graduate students. 

Similar to the teaching requirement, I would like to develop a diversity requirement for graduate students to increase their exposure to new people, places and ideas. The requirement would be flexible and up to the committee to determine appropriateness. This could include a community-based project, volunteering to a DE\&I oriented program, or a semester at a CG center or another university, , or even a semester . I also think a diversity of experience in graduate school would also benefit students and faculty. 


One way to accomplish this might be to set up a graduate student exchange program between universities in very different parts of the countryn with different demographics, traditions, farming practices etc. such as the South and Southwest. 
% I developed an infographic to describe the demographics of the participants, and am currently working to develop a pamphlet to encourage donors to join our effort



\section{Diversity, Inclusion and Equality in lab culture}
 % potential , where I discovered the difficulties they face with consistent funding. The graduate school SIPS has generously donated both funding and staff time for scheduling travel, and learned and 


Committment to diverssity in hiring

Long working hours in a windowless environment, equality in the little thngs matters, including ensuring that everyone has equally good chairs and equipment (e.g. monitors).




% DPW recruits people from underrepresented groups from around the United States and US territories that are interested in graduate school. Selected participants are invited to attend a fully-funded weekend at Cornell with the primary goal of empowering participants through workshops, campus tours, and familiarization with the application process. 

% I have actively engaged in learning about diversity, equity and inclusion as well as joined efforts to . 


%heterosexual white males and  % How can I help provide the  and opportunity to give a chance to those less fortunate?

% New Mexico, shocked by the paucity lack of Latin and native americans at Cornell. 


% I recognize the benefit of , as I come from a place with very different demographics.

% homogeneous. 
% While the plant breeding community has seen some progression in the balance of men and women in the field, the same is not necessarily true of different races and ethnicities. Indeed, Plant Breeding at Cornell is rather diverse from a global perspective; however, coming from New Mexico and a Hispanic-Serving Institution (NMSU), I was struck by the relative paucity of \emph{American} minority students. The plant breeding community suffers from a lack of these important cultural perspectives and I would like to help Cornell to be a leader in including these underrepresented groups at a representative level.





% In an effort to address the need for more diversity in STEM graduate programs, graduate students from the Ecology Evolutionary Biology, Entomology and School of Integrative Plant Sciences (SIPS) have joined forces to build the Diversity Preview Weekend (DPW). DPW recruits people from underrepresented groups from around the United States (and US territories) that are interested in graduate school, but have little access to what a graduate degree is like. DPW selects applicant based on merit and need, and pays for them to attend a weekend at Cornell consisting of workshops, tours, familiarization with the applications process and funding sources, as well as networking opportunities. 


% founded the Fundraising Chair for the DPW 2020 cycle. While 

I joined an initiative to reach out to donors outside the university, in which I developed an infographic to describe the demographics of the participants, and am currently working to develop a pamphlet to encourage donors to join our effort. I have also collaborated with another member of DPW to write an abstract detailing the effectiveness of the program, that we submitted for a talk at The Allied Genetics Conference (TAGC, Genetics Society of America, April 2020).

I hope to continue support of this important program through advocacy as a faculty member.  

Moving forward, I want to establish connections with the ithaca community 


A relatively new concept to me is the hidden curriculum, where students learn skills/behaviors Fwithout instruction. These cirricula can be a challenge. 

\section{Lab culture}


% American minorities often have a family history of hardship, and I believe they are an important part of our diverse community that is underrepresented in Plant Breeding at Cornell. 
% For example, there are graduate students from all over the globe, but I could not help but notice the scarcity of \emph{American} minorities.  %experienced or fled extreme hardship in their families past. 


%Notable exceptions include Cornell's Barbara McClintock, who studied maize genetics and transposable elements and who has since become an important figure for women in science. Unfortunately, she was discouraged by her advisor, R. A. Emerson, to pursue graduate degree in plant breeding, supposedly because he felt she would not be eligible for employment with such a degree. 


% Given the mixed history of agriculture and these groups, perhaps it is not surprising that interest in plant breeding is low. %  greater into higher standing? levels? proportions?

% inclusive 

I want to help establish community outreach initiatives to identify, engage and recruit minority high school students to the plant sciences. We should be targeting low-income school districts, especially in urban areas, where young people may be unfamiliar with the career opportunities in agriculture. These careers can provide a way for aspiring young people to make real differences in the lives of others. We also need to actively recruit students at the undergraduate level. I would like to establish collaborations with universities in the South and Southwest US to build channels to recruit Black, Latino and Native American students to the graduate school. 

The curriculum needs to be reflective of the breadth of diversity, such that students recognize a bit of themselves, their culture or contributions that were made by people they can relate to. All cultures have their own traditions and customs for distributing seed, selection and cultivation. These practices can often also be assigned as unique gender roles. It is important that we expose students to ideas outside of the US and western European seed systems. To incorporate these ideas into the curriculum, I want to explore ways to teach about plant domestication, maintenance of genetic variation, and trait selection from cultural and gender oriented perspectives. % through time and space. 

% As widespread agriculture was an integral part of most people's daily lives up until rather recently, 

%Interesting examples include the redistribution of new world crops to old world cultures, where many of the traditional foods of varying old world cultures include these new world crops. 

% Maize has a rich history of domestication and dissemination throughout the world, and different cultures use maize in many different ways.

Initially, I would like to use my connections in New Mexico to establish relationships with Puebloan farmers in New Mexico and Arizona to bring a perspective on their seed systems through guest lectures to introductory courses, such as PLBRG2010. This will cultivate classroom discussion on how these systems differ from modern US seed systems, and what the benefits and consequences of these systems are. Other unique seed systems, such as those in various parts of Africa, may also be introduced through students and researchers from those areas that collaborate with Cornell. %Native American seed systems tend to encourage maintenance of genetic diversity and unique traits, yet they are not adapted to the modern US agriculture system. 


The opportunity to purse higher education should be accessible to all people regardless of race, culture, ethnicity, gender, sexuality, ability, disability, religion, nationality, socioeconomic status, beliefs or geographic origin. Educators have a duty to provide curriculum and instruction in an inclusive setting that's reflective of our diverse communities, where each perspective contributes to a wealth of experience and knowledge. %I hope to contribute to the continued legacy of diversity, equity and inclusion at Cornell University. 

% different perspectives backgrounds experiences add to collaborate . multiple perspectives, , increasingly diverse representation, communities, 


% global community different flavors contribute in a meaningful way . 



Having recently been a grad studnet acutely aware that faculty may not understand to what degree a hidden cirriculum exists
Hidden cirriculum workshops for faculty


inclusive lab culture. 
team effort, 

heirarchical mentoring -> recent evidence that shows mentorship outside advisor most important. 
inclusinve teaching tactics


actively thinking about and learning DE\&I - > eg hidden cirriculum 

environment where people can make mistakes while learning and practicing without fear of retribution being ostricized. help learn how to conversation.
 
\end{document}
